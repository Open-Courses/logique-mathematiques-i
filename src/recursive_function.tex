\chapter{Fonctions récursives}

% Références :
%	- Boffa

\section{Historique}

Les mathématiques ont comme base le calcul. % Provenance du mot calcul
Au fil des années, les méthodes de calculs ont évoluées, et les mathématiques ont
évoluées en diverses branches, comme l'algèbre, l'analyse, les probabilités, les
statistiques. Le calcul est alors devenue une branche des mathématiques, appelée
l'arithmétique.

Au fur et à mesure des années, on a cherché des méthodes automatiques de calcul.
On a cherché à éviter de répéter, encore et encore, les mêmes calculs. On a
aussi également cherché à simplifier certains calculs. %Exemple des tables de log.

Bien que la notion de \textbf{calcul} nous semble évidente, nous avons besoin de
définir cette notion.


\section{Premières définitions}

Posons directement la définition de \textbf{fonctions récursives}:

\begin{definition}
	Une \textbf{fonction récursive} est une fonction calculable par un
	algorithme. C'est-à-dire qu'il existe un algorithme qui pour tout $a$ dans
	l'ensemble de départ de la fonction, calcule son image.
\end{definition}

Cette définition nous définit la classe des fonctions dont on est capable de
calculer grace à une méthode de calcul.

Il reste cependant à définir quel méthode de calcul nous utilisons, et bien sûr,
de définir de manière rigoureuse ce qu'est un algorithme.

D'abord, il nous faut poser ce qu'on entend par \textbf{calcul}. ?????

méthode de calcul ?????

Il existe plusieurs méthodes de calcul:

\begin{itemize}
	\item Machine de Turing
	\item $\lambda$-calcul
	\item Système de Post
	\item Algorithmes de Markov
\end{itemize}

\section{Notions de base}

Bien que nous n'ayons pas défini la notion \textit{d'algorithme}, nos
expériences en informatique nous donne une intuition. Un algorithme est une
suite finie d'instruction qui pour une certaine entrée, nous donne une sortie.

Nous supposerons par la suite que nous disposons d'une définition précise, et
définie mathématiquement, et dont notre intuition vérifie la définition.
Nous pourrons alors utiliser notre expérience pour donner des résultats.

\begin{definition}
	Soit $X$ un ensemble.

	On dit que $X$ est \textbf{un espace} s'il vérifie les conditions suivantes:

	\begin{enumerate}
		\item $X$ est infini
		\item $X \subset A^{*}$ où $A$ est un ensemble fini et $A^{*}$ est
			l'ensemble des suites finies qui ont comme composantes des éléments
			de $A$.
		\item Soit $x \in A^{*}$, alors il existe un algorithme booléen $A_{x}$ tel que
			$x \in X \equiv A_{x}(x)$ vrai.
	\end{enumerate}
\end{definition}

Nous allons alors supposer que certains ensembles soient des \textit{espaces}.
Les voici:

\begin{enumerate}
	\item $\naturel$ est un espace.
	\item Soient deux ensembles $X$ et $Y$ qui sont des espaces resp. sur $A^{*}
		$ et $B^{*}$, alors $X
		\cartprod Y$ est un espace.
	\item Soit $X$ un ensemble qui est un espace, alors l'ensemble des mots
		finis sur $X$ ie $X^{*}$, est un espace.
\end{enumerate}


