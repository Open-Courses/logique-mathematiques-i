\chapter{Fonctions récursives}

% Références :
%	- Boffa

\section{Historique}

Les mathématiques ont comme base le calcul. % Provenance du mot calcul
Au fil des années, les méthodes de calculs ont évoluées, et les mathématiques ont
évoluées en diverses branches, comme l'algèbre, l'analyse, les probabilités, les
statistiques. Le calcul est alors devenue une branche des mathématiques, appelée
l'arithmétique.

Au fur et à mesure des années, on a cherché des méthodes automatiques de calcul.
On a cherché à éviter de répéter, encore et encore, les mêmes calculs. On a
aussi également cherché à simplifier certains calculs. %Exemple des tables de log.

Bien que la notion de \textbf{calcul} nous semble évidente, nous avons besoin de
définir cette notion.


\section{Premières définitions}

Posons directement la définition de \textbf{fonctions récursives}:

\begin{definition}
	Une \textbf{fonction récursive} est une fonction calculable par un
	algorithme. C'est-à-dire qu'il existe un algorithme qui pour tout $a$ dans
	l'ensemble de départ de la fonction, calcule son image.
\end{definition}

Cette définition nous définit la classe des fonctions dont on est capable de
calculer grace à une méthode de calcul.

Il reste cependant à définir quel méthode de calcul nous utilisons, et bien sûr,
de définir de manière rigoureuse ce qu'est un algorithme.

D'abord, il nous faut poser ce qu'on entend par \textbf{calcul}. ?????

méthode de calcul ?????

Il existe plusieurs méthodes de calcul:

\begin{itemize}
	\item Machine de Turing
	\item $\lambda$-calcul
	\item Système de Post
	\item Algorithmes de Markov
\end{itemize}

\section{Notions de base}

Bien que nous n'ayons pas défini la notion \textit{d'algorithme}, nos
expériences en informatique nous donne une intuition. Un algorithme est une
suite finie d'instruction qui pour une certaine entrée, nous donne une sortie.

Nous supposerons par la suite que nous disposons d'une définition précise, et
définie mathématiquement, et dont notre intuition vérifie la définition.
Nous pourrons alors utiliser notre expérience pour donner des résultats.

\begin{definition}
	Soit $X$ un ensemble.

	On dit que $X$ est \textbf{un espace} s'il vérifie les conditions suivantes:

	\begin{enumerate}
		\item $X$ est infini
		\item $X \subset A^{*}$ où $A$ est un ensemble fini et $A^{*}$ est
			l'ensemble des suites finies qui ont comme composantes des éléments
			de $A$.
		\item Soit $x \in A^{*}$, alors il existe un algorithme booléen $A_{x}$ tel que
			$x \in X \equiv A_{x}(x)$ vrai.
	\end{enumerate}
\end{definition}

Nous allons alors supposer que certains ensembles soient des \textit{espaces}.
Les voici:

\begin{enumerate}
	\item $\naturel$ est un espace.
	\item Soient deux ensembles $X$ et $Y$ qui sont des espaces resp. sur $A^{*}
		$ et $B^{*}$, alors $X \cartprod Y$ est un espace sur $A^{*} \union
		B^{*} \union \GSset{\xi}$ où $\xi$ est un symbole n'appartenant à
		aucun des alphabets $A$ et $B$.
	\item Soit $X$ un ensemble qui est un espace, alors l'ensemble des mots
		finis sur $X$ ie $X^{*}$, est un espace.
\end{enumerate}

Comme nous faisons pour n'importe quelle structure sur un ensemble, nous allons
définir les morphismes entre ces espaces. Cependant, pour garder une intuition
informatique, nous allons appelé ces morphismes des \textbf{fonctions
récursives}. Les fonctions récursives doivent transporter la structure des
éléments, ce qui motive la définition suivante.

\begin{definition}
	Soient $X$ et $Y$ deux espaces.
	Une fonction $\GSfunction{f}{X}{Y}$ est dite \textbf{récursive} s'il existe
	un algorithme $A$ tel que pour tout $x \in X$, $A$ calcule $F(x)$.
\end{definition}

Rappelons que nous n'avons pas défini précisément la notion d'algorithme.
Cependant, nous avons supposé que nous en avons défini une, et que cette
définition recouvre toutes les opérations que nous pouvons réaliser dans un
langage de programmation.

Donnons quelques exemples de \textit{fonctions récursives}.

\begin{exemple}
	\begin{itemize}
		\item Prenons $A = \naturel \cartprod \naturel$, et $B = \naturel$.
			Alors l'addition usuelle ($\GSfunction{+}{\naturel \cartprod
			\naturel}{\naturel}$) et la multiplication usuelle
			($\GSfunction{.}{\naturel \cartprod \naturel}{\naturel}$) sont des
			fonctions récursives.
		\item Soit $X$ un espace, alors l'identité $\GSfunction{Id_{X}}{X}{X}$
			est une fonction récursive.
		\item Soient $X$ et $Y$ deux espaces. Alors les fonctions de projections
			sur $X$ et sur $Y$ sont des fonctions récursives.
		\item Soit $X$ un espace, et prenons $X^{*}$ l'ensemble des mots finis
			sur $X$.
			Soit $\GSfunction{l}{X^{*}}{\naturel}$ la fonction qui calcule la
			longueur du mot. Alors $l$ est une fonction récursive.
	\end{itemize}
\end{exemple}

Prenons maintenant deux fonctions récursives $f$ et $g$. Si nous faisons le
parallèle avec les structures de groupe ou d'espaces vectoriels, nous savons que
la composition que la composition de deux morphismes est encore un morphisme.
Nous obtenons alors le même résultat pour les fonctions récursives.

\begin{proposition}
	Soient $X$, $Y$ et $Z$ trois espaces.
	Soient $\GSfunction{f}{X}{Y}$ et $\GSfunction{g}{Y}{Z}$ deux fonctions
	récursives.

	Alors $\GSfunction{g \circ f}{X}{Z}$ est aussi une fonction récursive.
\end{proposition}

\ifdefined\outproof
\begin{proof}

\end{proof}
\fi

Nous pouvons alors de la même manière que dans les structures connues, définir
les \textit{isomorphismes} entre les espaces.
Rappelons que si $f$ est un morphisme bijectif, il n'est pas toujours vrai que
$f^{-1}$ soit aussi un morphisme: cela dépend des structures. En effet, pour les
structure de groupe, $f^{-1}$ est également un morphisme. Or, pour les
applications continues (qui sont les morphismes des structures topologiques),
ce n'est pas toujours vrai.

Rappelons que la notion de bijectivité ne demande aucune structure, celle-ci
n'est qu'ensembliste.

\begin{definition}
	Soient $X$ et $Y$ deux espaces.
	Soit $\GSfunction{f}{X}{Y}$ une fonction récursive bijective tel que
	$\GSfunction{f^{-1}}{Y}{X}$ est aussi récursive.
	Alors on dit que $f$ est \textbf{un isomorphisme}.
\end{definition}

Nous arrivons alors à un théorème très surprenant concernant les espaces.

Revenons aux structures algébriques que nous connaissons, comme celle de groupe.
Nous savons que pour deux groupes donnés, il n'existe pas nécessairement
d'isomorphismes entre les deux.
