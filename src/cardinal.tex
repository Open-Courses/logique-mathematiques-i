\chapter{Cardinaux}

% Faire les preuves avec les ordinaux, ou donner un lien vers l'article sur les
% ordinaux, et y faire les preuves.

\begin{definition}
	Soit $A$ et $B$ deux ensembles.
	On dit que $A$ et $B$ \textbf{ont même cardinal} s'il existe une fonction
	$\GSfunction{f}{A}{B}$ bijective.
\end{definition}

\begin{definition}
	Le cardinal d'un ensemble $A$, noté $\card{A}$, est le plus petit ordinal
	$\alpha$ tel que $\alpha$ est en bijection avec $A$.
\end{definition}

\begin{definition}
	Soit $A$ et $B$ deux ensembles. On dit que \textbf{le cardinal de $A$ est
	plus petit (resp. plus grand) que le cardinal de $B$}, noté $\card{A} \leq
	\card{B}$ (resp $\card{A} \geq \card{B}$) s'il existe une fonction
	$\GSfunction{f}{A}{B}$ injective (resp. surjective).
\end{definition}

\begin{theorem} [Cantor-Schroder-Bernstein]
	Soit $A$ et $B$ deux ensembles. On a:

	$(\card{A} \leq \card{B}$ et $\card{A} \geq \card{B}) \equiv \card{A} =
	\card{B}$.
	\label{thm:csb}
\end{theorem}
\begin{definition}
	Un ensemble $A$ est \textbf{dénombrable} s'il a le même cardinal que
	$\naturel$.
\end{definition}

\begin{proposition}
	Soit $A$ un ensemble infini. Alors contient un ensemble dénombrable.
\end{proposition}

\begin{proof}
	
\end{proof}

\begin{proposition}
	\label{prop:infinite_set_partition}
	Soit $A$ un ensemble infini. Alors il existe une partition de $A$ d'ensemble
	dénombrable.
\end{proposition}

\begin{proof}

\end{proof}

\begin{theorem}
	\label{thm:cartesian_product_denombrable_infini_infini}
	Soit $A$ et $D$ deux ensembles infinis tel que $D$ est \textit{dénombrable}.
	Alors $\card{A \cartprod D} = \card{A}$.
\end{theorem}

\begin{proof}

\end{proof}

En particulier, on a donc que $\card{\naturel \cartprod \naturel} = \card{\naturel}$

\begin{corollary}
	\label{cor:union_infini_set}
	Soit $A$ et $B$ deux ensembles tel que $B$ est infini et $\card{A} \leq \card{B}$.
	Alors $\card{A \union B} = \card{B}$.
\end{corollary}

\begin{proof}
	
\end{proof}

\begin{theorem}
	Soit $A$ un ensemble infini. Alors $\card{A \cartprod A} = \card{A}$.
	\label{thm:card_cart_product_infini}
\end{theorem}

\begin{proof}
	
\end{proof}

\begin{corollary}
	Soit $A$ un ensemble infini et $n \in \naturel$. Alors $\card{A^{n}} =
	\card{A}$.
\end{corollary}

\begin{proof}
	
\end{proof}

\begin{corollary}
	Soit $A$ un ensemble infini et $\Omega$ l'ensemble des parties
	\textit{finies} de $A$. Alors $\card{A \cartprod \Omega} = \card{A}$.
\end{corollary}

\begin{proof}
	
\end{proof}

\begin{corollary}
	Soit $A$ un ensemble infini et $\Omega$ l'ensemble des suites
	\textit{finies} de $A$. Alors $\card{A \cartprod \Omega} = \card{A}$.
\end{corollary}

\begin{proof}
	
\end{proof}

